%
% FECHA: 2020-11-12
% AUTOR: Julio Alejandro Santos Corona
% CORREO: jualesac@yahoo.com
% TÍTULO: ajax.tex
%
% Descripción: Documentación de la clase AJAX.
%
\documentclass[10pt]{article}
\usepackage[spanish]{babel}
\usepackage[utf8]{inputenc}
\usepackage{listings}
\usepackage{anysize}
\usepackage{colortbl}
\title{AJAX v.1.2}
\author{Julio Alejandro Santos Corona}
\date{12 de noviembre de 2020}

\marginsize{2.5cm}{2.5cm}{0cm}{1.5cm}

\definecolor{comment}{rgb}{0.5, 0.5, 0.5}
\definecolor{background}{rgb}{0.21, 0.24, 0.25}

\begin{document}
\lstdefinelanguage{JavaScript}{
	morekeywords={
		new,
		function,
		var,
		let,
		get,
		post,
		put,
		delete,
		route,
		use
	},
	sensitive=false,
	morecomment=[l][\color{comment}]{//},
	morecomment=[s]{/*}{*/}
}

\lstset{breaklines=true, tabsize=4, language=JavaScript}
\lstset{basicstyle=\small ,numbers=left, numberstyle=\small, stepnumber=1, numbersep=-12pt, backgroundcolor=\color{white}, frame=leftline}

\maketitle

La clase es una simplificación para realizar peticiones ajax, para ello utiliza la api web fetch de javascript.

\section{Instancia}

Para crear una instancia basta con crear un objeto a través del constructor de la función AJAX.main; a este se le puede pasar un booleano (default: true) que definirá el comportamiento de las rutas configuradas:
\\
\begin{lstlisting}
	var ajax = new AJAX.main; five
	ajax.route (``localhost:9001/index.js"); //Entrada al backend
\end{lstlisting}

\section{Peticiones}
La clase cuenta con métodos para realizar peticiones de forma fácil y con una estructura similar.

\subsection{GET, POST, PUT, DELETE}

Los verbos más utilizados en un api rest y la estructura de uso es similar para todos los casos, haciendo una distinción en el método get que el argumento de valores es opcional y dichos valores son transmitidos en la URL.
\\\\
\begin{tabular}{|m{1.8cm}|m{1.5cm}|m{1.5cm}|m{9.4cm}|}
	\hline
	\rowcolor{black}\textcolor{white}{Argumento} & \textcolor{white}{Tipo} & \textcolor{white}{Opcional} & \textcolor{white}{Descripción} \\
	\hline
	callback & function & No & Función a ejecutar luego de la resolución de la petición, recibe dos argumentos, res y status. Res contiene los datos en formato json y status el estado http de respuesta. \\
	\hline
	body & object & No & Objeto js que contiene los valores que serán pasados a través del body; en caso del método get este argumento es opcional y los valores son transmitidos en la URL. \\
	\hline
	url & string & No & Ruta a la que se hará la petición, el comportamiento está dado por el modo en que se creó la instancia. \\
	\hline
	middle & boolean & Si & Permite o niega la ejecución de un middleware. (Default: true) \\
	\hline
\end{tabular}
\\\\\\
Ejemplo: 
\begin{lstlisting}
	//Peticiones get con y sin valores
	ajax.get (function (res, status) {
		//codigo
	}, { var1: valor1 }, "/APP1");
	
	ajax.get (function (res, status) {
		//codigo
	}, "/APP1");
	
	ajax.post (function (res, status) {
		//codigo
	}, { var1: valor1 }, "/APP");
	
	ajax.put (function (res, status) {
		//codigo
	}, {}, "/APP");
	
	ajax.delete (function (res, status) {
		//codigo
	}, {}, "/APP", false);
\end{lstlisting}

\subsection{FORM y TEXT}

La clase cuenta con dos métodos extra que tienen la finalidad de realizar peticiones especiales.
\\\\
El método \textbf{form} se puede utilizar cuando se cuente con una vista en la que exista una estructura html form, la petición se enviará siempre a través de POST y el body serán los objetos input, textarea, etc. contenidos en el.
\\\\
\begin{tabular}{|m{1.8cm}|m{2cm}|m{1.5cm}|m{9cm}|}
	\hline
	\rowcolor{black}\textcolor{white}{Argumento} & \textcolor{white}{Tipo} & \textcolor{white}{Opcional} & \textcolor{white}{Descripción} \\
	\hline
	callback & function & No & Función a ejecutar luego de la resolución de la petición, recibe dos argumentos, res y status. \\
	\hline
	form & object$/$string & No & El argumento puede recibir un objeto FormData o el id de la etiqueta form. \\
	\hline
	url & string & No & Ruta a la que se hará la petición. \\
	\hline
	middle & boolean & Si & Permite o niega la ejecución de un middleware. (Default: true) \\
	\hline
\end{tabular}
\\\\
\noindent
\footnotesize{HTML} \normalsize{}
\lstset{language=HTML}
\begin{lstlisting}
	<form id="formulary">
		<input type="text" name="name">
		<input type="number" name="age">
	</form>
\end{lstlisting}

\noindent
\footnotesize{JS} \normalsize{}
\lstset{language=JavaScript}
\begin{lstlisting}
	var ajax = new AJAX.main;
	ajax.route ("localhost:9001/index.js");
	
	let form = new FormData (document.getElementById ("formulary"));
	
	ajax.form (function (res, status) {
		//codigo
	}, form, "/FORMULARIO");
	
	//El metodo tambien permite pasar solo el id del formulario
	ajax.form (function (res, status) {
		//codigo
	}, "formulary", "/FORMULARIO");
\end{lstlisting}
\noindent
El método \textbf{text} se utilizará para obtener el contenido de un archivo, para éste método no se ejecuta middleware, su definición es la siguiente:
\\\\
\begin{tabular}{|m{1.8cm}|m{2cm}|m{1.5cm}|m{9cm}|}
	\hline
	\rowcolor{black}\textcolor{white}{Argumento} & \textcolor{white}{Tipo} & \textcolor{white}{Opcional} & \textcolor{white}{Descripción} \\
	\hline
	callback & function & No & Función a ejecutar luego de la resolución de la petición, recibe dos argumentos, res y status. \\
	\hline
	url & string & No & Ruta al archivo. \\
	\hline
	content & string & Si & Header Content-Type, por default TEXT/PLAIN \\
	\hline
\end{tabular}
\\\\\\
Ejemplo:

\begin{lstlisting}
	ajax.text (function (text, status) {
		//codigo
	}, "/VIEWS/view.html", "TEXT/HTML");
\end{lstlisting}

\newpage

\section{Manejo de rutas}

El comportamiento de las rutas se define durante el constructor, por default URLStatic tiene un valor true.
\\\\
Las rutas se configuran a través del método \textbf{route} y \textbf{use}. La recomendación es configurar a través de \textbf{route} la ubicación del punto de entrada al backend y posteriormente definir la entrada de app con use.
\\\\
Cuando URLStatic es true la ruta se forma con \textbf{route} + \textbf{use} + \textbf{url} de lo contrario será \textbf{use} + \textbf{url}.
\\\\
Ejemplo:
\begin{lstlisting}
	var ajax = new AJAX.main;
	
	ajax.route ("localhost:9001"); //Configuracion del punto de entrada a backend
	ajax.use ("/APP"); //Configuracion del punto de entrada de app
	
	//Peticion POST a ruta localhost:9001/APP/CONFIG?idUser=193
	ajax.post (function (res, status) {
		//codigo
	}, { mail: "user@server.com" }, "/CONFIG?idUser=193");
\end{lstlisting}

\section{Otros}

\subsection{Manejador de errores}

La clase por default cuenta con un manejo de errores estándar que puede ser sustituído y adaptado a las necesidades del aplicativo en caso de existir un problema con la petición.
\\
\begin{lstlisting}
	ajax.setErrors (function (err) {
		if (err.status = 404) {
			console.log ({ type: "Error", error: "err" });
		}
	});
\end{lstlisting}

\subsection{Middleware}

Durante las peticiones es posible configurar un middleware que se ejecutará antes de la función resolutora, dicho middleware recibe como argumentos la propia respuesta de la petición y el estatus de la misma.
\\
\begin{lstlisting}
	ajax.setMiddleware (function (res, status) {
		//codigo
	});
\end{lstlisting}

\subsection{Headers}

Es posible, en caso de requerirlo, configurar las cabeceras que se incluirán en las peticiones a través del objeto \textbf{Headers} y del que se tiene acceso directo en ajax.\textbf{header}.

\end{document}